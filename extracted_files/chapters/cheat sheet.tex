\chapter{Introduction}
\label{introduction}
\section{	Fundamentals of semiconductors under the lens of physics }

table 
\begin{table}[h]
\centering
\caption{literature review of semiconductor devices }
\begin{tabular}{|c|c|}
\hline
1874 & Braun discovers asymmetric\\ &conduction in semiconductor metal contacts\\
\end{tabular}
\label{tab:1}
\end{table}
equation
\begin{equation}
\label{eq:1}
    y = \sum_{i=0}^{N-1}a_ix + b_i.
\end{equation}


bullet and number :
classifications are grouped as :
\begin{enumerate}
    \item Elemental semiconductors: Si, Ge in which all sites are occupied by same atom.
    \item Compound semiconductors (1:1 stoichimetry):
    \begin{itemize}
    \item III-V: Antimonides : AlSb, GaSb, InSb
    \item Aresides: AlAs, GaAs, InAs
    \item Phoshides: AlP, GaP, InP
    \item Nitrides: AlN, GaN, InN
    \item IV-IV: SiC
    \item II-VI: CdS, ZnS,ZnSe,ZnTe,MgS…
    \item II-VI CdSe,CdTe
    \item IV-VI PbSe, PbS,PbTe…
    \item Some Metal Oxydes TiO2, ZnO, CuO,InO….
\item Alloys 
    
\end{itemize}
\end{enumerate}

bullet
\begin{itemize}
    \item hiviiiiiiiii
    \item mojeeeeeee
\end{itemize}
bold
\textbf{karlsruhe}

citation
~\cite{ref1}.  

\begin{equation}
\label{eq4}
   DOS=\frac{1}{2\Pi^3 }\left[ \frac{4\Pi k^2}{dE/dk}\right]=\frac{1}{2\Pi^2 }  \sqrt{2Em^3/ \overline{h}^3}
\end{equation}
italic :
\textit{accident}




















\section{}\label{sect:1_1}  

Text of Section~\ref{sect:1_1}, in Chapter~\ref{chapt:1}. 
A figure is reported in Figure~\ref{fig:1}.

\begin{figure}[h!]
\centering
\includegraphics[width=0.5\textwidth]{figures/example.png}
\caption[Example of caption.]{Example of caption.\label{fig:1}}
\end{figure}

\section{Section 2 of chapter 1}\label{sect:1_2} 

\subsection{Subsection of section 2}\label{subsect:1_2_1} 
This is a subsection.
An example of table is in Table~\ref{tab:1}. 
\begin{table}
\centering
\begin{tabular}{c|c}
\toprule
field$1$ & field$2$ \\ \midrule
value$1$ & value$2$ \\
\bottomrule
\end{tabular}
\caption[Example of caption for the table.]{Example of caption for the table.}
\label{tab:1}
\end{table}






\section{Section 3 of chapter 1}\label{sect:1_3} 
This is an example of equation: Equation~\ref{eq:1}.
\begin{equation}
\label{eq:1}
    y = \sum_{i=0}^{N-1}a_ix + b_i.
\end{equation}

\section{Section 4 of chapter 1}\label{sect:1_4} 
Some examples of references. 
This is a book reference~\cite{book1}.  
This is an article reference~\cite{article1}.  
This is a conference reference~\cite{conference1}.  
This is a reference for an online resource~\cite{online1}.  
This is a reference for a standard~\cite{standard1}.  

\begin{figure}[h!]
\centering
\includegraphics[width=0.5\textwidth]{figures/example.png}
\caption[Example of caption.]{Example of caption.\label{fig:1}}
\end{figure}


ref to chapter, fig , equ , table 
inside aculad is lable ~\ref{fig:1} 

tableeeeeeeeeeeee
\begin{table}[h]
\centering
\begin{tabular}{c|c|c|c|c}
\toprule
field$1$ & field$2$ \\ \midrule
value$1$ & value$2$ \\
\bottomrule
\end{tabular}
\caption[Example of caption for the table.]{Example of caption for the table.}
\label{tab:1}
\end{table}