\chapter{results}
\label{result}
The output curve of a transistor shows the different operating regions.
To obtain the output curve, the gate-source voltage is fixed at a certain value and the
drain-source current is plotted against the drain-source voltage. The saturation current
of a transistor is ideally dependent on the gate-source voltage, rather than on the drain-source voltage. In the case of the transfer curve, the drain-source voltage
is kept constant while the drain-source current is plotted against the gate-source voltage. Thereby, the drain-source current is described by the following equation, well known from literature and implemented in the Shichman-Hodges transistor model (Level 1
model)


\section{Integrated Circuit}
 Integrated Circuits high degree of TFT integration is necessary for manufacturing flexible circuits for future applications. In accordance with this requirement,  ... oxide  TFTs  have  been  applied  to  a    metal  oxide  semiconductor  (CMOS)  circuit
 
\subsection{ }
\label{}
previous abstract
Through the intriguing characteristics such as being additive, contactless, compatible over variety of substrates and large area applications, printed electronics (PE) have attracted tremendous attention to enable the design space exploration of microprocessors in large scale integrated circuits, by which the ultra-low cost margin domain can be attained. To reach a fundamental building block of those microprocessors with arithmetical functionality for example in computers or neuromorphic applications, the objectives of this research cover two major themes in the domain of flexible and printed electronics: preparation and fabrication techniques of functional materials. 
Firstly, the fabricated standard unit cell EGFET has been discussed. For that aim, using a proper material is a crucial challenge, for that reason in the following introduction chapter the rational choice based on physical and electrical properties of functional materials for stacks of individual transistor containing: dielectric, conductive layers and electrode material for a single partially-inkjet printed transistor has been discussed. Furthermore, the improvement of circuit’s design in the case of complex circuits by avoiding unwanted crossings and intersections by a insulating material was possible. Subsequently, an improvement of the circuit design performed using k-layout software for reducing number of transistors which can ultimately provide configurations made by printed electronics with more finial gain. Here this has been applied to the passive logic gates. The mentioned layers have been printed using top-gated field-effect transistor with in-plane structure(ref) and $In_2O_3$, CSPE poly(vinyl alcohol) (PVA) dissolved in dimethyl sulfoxide (DMSO), poly (3,4- ethylenedioxythiophene) doped with polystyrene sulfonate (PEDOT:PSS) as channel semiconducting layer, electrolyte layer and top-gate layer has been used respectively. 
For the fulfilment of successful operation of the state-of-the-art adder structure, the optimized annealing temperature for metal oxide inorganic semiconductor $In_2O_3$ precures based, has found to be set at 300°C. Later led to better performance for individual transistor by shifting the threshold voltage towards positive range which will be fully discussed in chapter[???] - to the best of my knowledge. Moreover, according to the fact that in printed electronics, a circuit can be manufactured on-demand at the point-of-use, here in this work for the flexibility application (1), further trials have been done also over using nano particle based $In_2O_3$ semiconductor material for channel of individual transistor.
Also, In order to facilitate the adder circuit, basic logic units including Inverter, NAND, NOR and most importantly XOR gate has been fabricated and tested.  
The inkjet-printed EGFETs show promising properties, such as excellent ON/OFF ratio that can reach …., no obvious degradation after ??? days and low operation voltage around 2??V for all devices.
Fabricated individual transistors using In2O3 precursor-based ink and with$ W/L$ of $200 [µm]/20 [µm]$ has yield of 92\%. The value for some parameters related to physical and electrical properties of used material has been reported as follows: Sub threshold slop: 73.1 $mV/dec$, Ion: 1.313 mA and most importantly positive value for $V_{th}$ (threshold voltage) $V_{th}: 0.1V$. Compared to previous work with annealing at 400°C in which the threshold voltage parameter reached negative value of $V_{th}$: $-0.2$, in this work a positive threshold voltage gives rise to rail-to-rail swing of the transistor.  In the following, the fabrication of mono-type (n-type) logic gates such as Inverter, NAND, NOR and XOR with 100\% yield factor and rise time: $1.5 [ms]$ and fall time: $0.8 [ms]$ has been performed. The half adder and subsequently the state-of-the-art adder as fundamental building block of the ALU processor has been successfully built. Furthermore, pass logic gates has been fabricated by which the number of transistors needed for making XOR gate reduced from 8 to 5. 
To be completed                      
